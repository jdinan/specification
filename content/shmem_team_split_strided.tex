\apisummary{
Create a new \openshmem team from a subset of the existing parent team \acp{PE},
where the subset is defined by the
\ac{PE} triplet (\VAR{start}, \VAR{stride}, and \VAR{size}) supplied to the routine.}

\begin{apidefinition}

\begin{Csynopsis}
int @\FuncDecl{shmem\_team\_split\_strided}@(shmem_team_t parent_team, int start, int stride, int size,
    const shmem_team_config_t *config, long config_mask, shmem_team_t *new_team);
\end{Csynopsis}

\begin{apiarguments}
\apiargument{IN}{parent\_team}{An \openshmem team.}

\apiargument{IN}{start}{The first \acs{PE} number of the subset of \acp{PE} from the parent team that will form the new team. If the stride is less than zero, the first \acs{PE} number is the highest \acs{PE} of the parent team; if it is  greater than zero, it is the lowest; if the stride is zero, it is the starting \acs{PE}.}

\apiargument{IN}{stride}{The stride between team \ac{PE}
numbers in the parent team that comprise the subset of \acp{PE} that will form
the new team.}

\apiargument{IN}{size}{The number of \acp{PE} from the parent team in the subset
of \acp{PE} that will form the new team. \VAR{size} must be a positive integer.}

\apiargument{IN}{config}{
  A pointer to the configuration parameters for the new team.}

\apiargument{IN}{config\_mask}{
  The bitwise mask representing the set of configuration parameters to use
  from \VAR{config}.}

\apiargument{OUT}{new\_team}{An \openshmem team handle. Upon successful creation, it
references an \openshmem team that contains the subset of all \acp{PE} in the
parent team specified by the \ac{PE} triplet provided.}

\end{apiarguments}

\apidescription{
The \FUNC{shmem\_team\_split\_strided} routine is a collective routine.
It creates a new \openshmem team from an existing parent team,
where the \ac{PE} subset of the resulting team is defined by the triplet of arguments
(\VAR{start}, \VAR{stride}, \VAR{size}).
A valid triplet is one such that:
\begin{equation*}
  start + stride \cdot i \in \mathbb{Z}_{N-1}
  \;
  \forall
  \;
  i \in \mathbb{Z}_{size-1}
\end{equation*}
where $\mathbb{Z}$ is the set of natural numbers ($0, 1, \dots$), $N$ is the
number of \acp{PE} in the parent team, $size$ is a positive number indicating
the number of \acp{PE} in the new team, and $stride$ is an integer.
The index $i$ specifies the number of the given PE in the new team.
When $stride$ is greater than zero, PEs in the new team remain in the same
relative order as in the parent team.
When $stride$ is less than zero, PEs in the new team are in \textit{reverse}
relative order with respect to the parent team.
If a $stride$ value equal to 0 is passed to \FUNC{shmem\_team\_split\_strided},
then the $size$ argument passed must be 1, or the behavior is undefined.
If the triplet provided to \FUNC{shmem\_team\_split\_strided} implies a
wrap-around sequence, the input is considered invalid and the behavior is
undefined.
In other words, when $stride$ is nonzero, a newly created team must only
include \acp{PE} whose subsequent parent \ac{PE} values are either all
increasing (for positive $stride$) or all decreasing (for negative
$stride$).
That is, \textit{wrap-around} with respect to the parent team's \ac{PE} values
is not permitted.
For example, given a parent team with a size of 8 \acp{PE}, a call to 
\FUNC{shmem\_team\_split\_strided} with the following arguments would
be invalid: $start$ equal to 3, $stride$ equal to 3, and $size$ equal to 3.

This routine must be called by all \acp{PE} in the parent team.
All \acp{PE} must provide the same values for the \ac{PE} triplet.
On successful creation of the new team:

\begin{itemize}
\item The \VAR{new\_team} handle will reference a valid team for the
  subset of \acp{PE} in the parent team that are members of the new team.
\item Those \acp{PE} in the parent team that are not members of the new team
  will have \VAR{new\_team} assigned to \LibConstRef{SHMEM\_TEAM\_INVALID}.
\item \FUNC{shmem\_team\_split\_strided} will return zero to all
  \acp{PE} in the parent team.
\end{itemize}

If the new team cannot be created or an invalid \ac{PE} triplet is provided,
then \VAR{new\_team} will be assigned the value \LibConstRef{SHMEM\_TEAM\_INVALID} and
\FUNC{shmem\_team\_split\_strided} will return a nonzero value on all
\acp{PE} in the parent team.

The \VAR{config} argument specifies team configuration parameters, which are
described in Section~\ref{subsec:shmem_team_config_t}.

The \VAR{config\_mask} argument is a bitwise mask representing the set of
configuration parameters to use from \VAR{config}.
A \VAR{config\_mask} value of \CONST{0} indicates that the team
should be created with the default values for all configuration parameters.
See Section~\ref{subsec:shmem_team_config_t} for field mask names and
default configuration parameters.

If \VAR{parent\_team}
compares equal to \LibConstRef{SHMEM\_TEAM\_INVALID}, then no new team
will be created, \VAR{new\_team} will be assigned the value
\LibConstRef{SHMEM\_TEAM\_INVALID}, and \FUNC{shmem\_team\_split\_strided} will
return a nonzero value.  If \VAR{parent\_team} is otherwise invalid, the
behavior is undefined.
}

\apireturnvalues{
  Zero on successful creation of \VAR{new\_team}; otherwise, nonzero.
}

\apinotes{
  The \FUNC{shmem\_team\_split\_strided} operation can take any integer value
  \VAR{stride} argument.

  See the description of team handles and predefined teams in
  Section~\ref{subsec:team} for more information about team handle semantics and usage.
}

\begin{apiexamples}

    \apicexample
    {The following example demonstrates the use of strided split in a
    \Cstd[11] program. The program creates a new team of all even number
    \acp{PE} from the world team, then retrieves the \ac{PE} number and
    team size on all \acp{PE} that are members of the new team.}
    {./example_code/shmem_team_split_strided.c}
    {}

\end{apiexamples}

\end{apidefinition}
