The \acp{PE} in an \openshmem program communicate using either
point-to-point routines---such as \ac{RMA} and \ac{AMO} routines---which specify the \ac{PE} number of the target
\ac{PE}, or collective routines, which operate over a set of \acp{PE}.
In \openshmem, teams allow programs to group a set of \acp{PE} for
communication.
Team-based collective communications operate across all the \acp{PE}
in a valid team.
Point-to-point communication can make use of team-relative \ac{PE}
numbering through team-based contexts (see Section~\ref{sec:ctx}) or
\ac{PE} number translation.

\subsubsection*{Predefined and Program-Defined Teams}

An \openshmem team may be predefined (i.e., provided by the \openshmem
library) or defined by the \openshmem program.
A program-defined team is created by ``splitting'' a parent team into
one or more new teams---each with some subset of \acp{PE} of the
parent team---via one of the \FUNC{shmem\_team\_split\_*} routines.

All predefined teams are valid for the duration of the \openshmem
portion of an application.
Any team successfully created by a \FUNC{shmem\_team\_split\_*}
routine is valid until it is destroyed.
All valid teams have a least one member.

\subsubsection*{Team Handles}

A ``team handle'' is an opaque object with type \CTYPE{shmem\_team\_t}
that is used to reference a team.
Team handles are not remotely accessible objects.
The predefined teams may be accessed via the team handles listed in
Section~\ref{subsec:library_handles}.

\openshmem communication routines that do not accept a team handle
argument operate on the default team, which may be accessed through
the \LibHandleRef{SHMEM\_TEAM\_WORLD} handle.
The default team encompasses the set of all \acp{PE} in the \openshmem
program, and a \ac{PE} number in the default team is the same as the
value returned by \FUNC{shmem\_my\_pe}.

A team handle may be initialized to or assigned the value
\CONST{SHMEM\_TEAM\_INVALID} to indicate that handle does not
reference a valid team.
When managed in this way, applications can use an equality comparison
to test whether a given team handle references a valid team.

\subsubsection*{Thread Safety}

When it is allowed by the threading model provided by the OpenSHMEM
library, a team may be used concurrently in non-collective operations
(e.g., \FUNC{shmem\_team\_my\_pe}) by multiple threads within the
\ac{PE} where it was created.
For collective operations, a team may not be used concurrently by
multiple threads in the same \ac{PE}.

\subsubsection*{Collective Ordering}

In \openshmem, a team object encapsulates resources used to communicate
between \acp{PE} in collective operations. When calling multiple subsequent
collective operations on a team, the collective operations---along with any
relevant team based resources---are matched across the \acp{PE} in the team
based on ordering of collective routine calls. It is the responsibility
of the \openshmem program to ensure the same ordering of collective routine calls
across all \acp{PE} in a team.

A full discussion of collective semantics follows in Section~\ref{subsec:coll}.

\subsubsection*{Team Creation}

Team creation is a collective operation on the parent team object. New teams
result from a \FUNC{shmem\_team\_split\_*} routine, which takes a parent team
and other arguments and produces new teams that are a subset of the parent
team. All \acp{PE} in a parent team must participate in a split operation
to create new teams. If a \ac{PE} from the parent team is not a member of any
resulting new teams, it will receive a value of \CONST{SHMEM\_TEAM\_INVALID}
as the value for the new team handle.

Teams that are created by a \FUNC{shmem\_team\_split\_*} routine may be
provided a configuration argument that specifies attributes of each new team.
This configuration argument is of type \CTYPE{shmem\_team\_config\_t}, which
is detailed further in Section~\ref{subsec:shmem_team_config_t}.

\acp{PE} in a newly created teams are consecutively numbered with starting with
\ac{PE} number 0. \acp{PE} are always ordered by the existing \ac{PE} number that
would be returned by the \FUNC{shmem\_team\_my\_pe} routine called on the parent team. Team relative \ac{PE}
numbers can be used for point-to-point operations through team-based
contexts (see Section~\ref{sec:ctx}) or using the translation routine
\FUNC{shmem\_team\_translate\_pe}.

As with any collective routine on a team, the program must ensure that there
are no simultaneous split operations occurring on the same parent team on a
given \ac{PE}, i.e. in separate threads.

As with any collective routine on a team, team creation is matched across PEs based
on ordering. So, team creation events must occur in the same order on all \acp{PE}
in the parent team.

Upon completion of a team creation operation, the parent and any resulting child teams
will be immediately usable for any team-based operations, including creating new child
teams, without any intervening synchronization.
