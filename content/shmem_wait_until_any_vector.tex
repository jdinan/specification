\apisummary{
    Wait on an array of variables on the local \ac{PE} until any one variable
    meets its specified wait condition.
}

\begin{apidefinition}

\begin{C11synopsis}
size_t @\FuncDecl{shmem\_wait\_until\_any\_vector}@(TYPE *ivars, size_t nelems, const int *status, int cmp,
    const TYPE *cmp_values);
\end{C11synopsis}
where \TYPE{} is one of the standard \ac{AMO} types specified by
Table \ref{stdamotypes}.

\begin{Csynopsis}
size_t @\FuncDecl{shmem\_\FuncParam{TYPENAME}\_wait\_until\_any\_vector}@(TYPE *ivars, size_t nelems, const int *status,
    int cmp, const TYPE *cmp_values);
\end{Csynopsis}
where \TYPE{} is one of the standard \ac{AMO} types and has a
corresponding \TYPENAME{} specified by Table~\ref{stdamotypes}.

\begin{apiarguments}

  \apiargument{IN}{ivars}{Symmetric address of an array of remotely accessible data
    objects.
    The type of \VAR{ivars} should match that implied in the SYNOPSIS section.}
  \apiargument{IN}{nelems}{The number of elements in the \VAR{ivars} array.}
  \apiargument{IN}{status}{Local address of an optional mask array of length \VAR{nelems}
    that indicates which elements in \VAR{ivars} are excluded from the wait set.}
  \apiargument{IN}{cmp}{A comparison operator from Table~\ref{p2p-consts} that
    compares elements of \VAR{ivars} with elements of \VAR{cmp\_values}.}
  \apiargument{IN}{cmp\_values}{Local address of an array of length \VAR{nelems}
    containing values to be compared with the respective objects in \VAR{ivars}.
    The type of \VAR{cmp\_values} should match that implied in the SYNOPSIS section.}

\end{apiarguments}

\apidescription{
    The \FUNC{shmem\_wait\_until\_any\_vector} routine waits until any one
    entry in the wait set specified by \VAR{ivars} and \VAR{status} satisfies
    the wait condition at the calling \ac{PE}.  The \VAR{ivars} objects at the
    calling \ac{PE} may be updated by an \ac{AMO} performed by a thread located
    within the calling \ac{PE} or within another \ac{PE}.
    This routine compares
    each element of the \VAR{ivars} array in the wait set with each respective
    value in \VAR{cmp\_values} according to the comparison operator \VAR{cmp}
    at the calling \ac{PE}.  The order in which these elements are waited upon
    is unspecified.  If an entry $i$ in \VAR{ivars} within the wait set
    satisfies the wait condition, a series of calls to
    \FUNC{shmem\_wait\_until\_any\_vector} must eventually return $i$.

    The optional \VAR{status} is a mask array of length \VAR{nelems} where each
    element corresponds to the respective element in \VAR{ivars} and indicates
    whether the element is excluded from the wait set.  Elements of
    \VAR{status} set to 0 will be included in the wait set, and elements set to
    a nonzero value will be ignored.  If all elements in \VAR{status} are nonzero or
    \VAR{nelems} is 0, the wait set is empty and this routine returns
    \CONST{SIZE\_MAX}.  If \VAR{status} is a null pointer, it is ignored and
    all elements in \VAR{ivars} are included in the wait set.  The \VAR{ivars}
    and \VAR{status} arrays must not overlap in memory.

    Implementations must ensure that \FUNC{shmem\_wait\_until\_any\_vector}
    does not return before the update of the memory indicated by \VAR{ivars} is
    fully complete.
}

\apireturnvalues{
    \FUNC{shmem\_wait\_until\_any\_vector} returns the index of an element in the
    \VAR{ivars} array that satisfies the wait condition. If the wait set is
    empty, this routine returns \CONST{SIZE\_MAX}.
}

\begin{apiexamples}
  \apicexample
      {The following \Cstd[11] example demonstrates the use of
      \FUNC{shmem\_wait\_until\_any\_vector} to wait on values that differ
      between even \acp{PE} and odd \acp{PE}.}
      {./example_code/shmem_wait_until_any_vector.c}
      {}
\end{apiexamples}

\end{apidefinition}
