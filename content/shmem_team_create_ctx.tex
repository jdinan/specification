\apisummary{
  Create a communication context from a team.
}

\begin{apidefinition}

\begin{Csynopsis}
int @\FuncDecl{shmem\_team\_create\_ctx}@(shmem_team_t team, long options, shmem_ctx_t *ctx);
\end{Csynopsis}

\begin{apiarguments}
  \apiargument{IN}{team}{A handle to the specified \ac{PE} team.}
  \apiargument{IN}{options}{
    The set of options requested for the given context.
    Multiple options may be requested by combining them with a bitwise OR
    operation; otherwise, \CONST{0} can be given if no options are requested.}
  \apiargument{OUT}{ctx}{A handle to the newly created context.}
\end{apiarguments}

\apidescription{
  The \FUNC{shmem\_team\_create\_ctx} routine creates a new communication
  context and returns its handle through the \VAR{ctx} argument.
  This context is created from the team specified by the \VAR{team} argument;
  however, the context creation operation is not collective.

  In addition to the team, the \FUNC{shmem\_team\_create\_ctx} routine accepts
  the same arguments and provides all the same return conditions as the
  \FUNC{shmem\_ctx\_create} routine.

  The \FUNC{shmem\_team\_create\_ctx} routine may be called any number of times
  to create multiple simultaneously existing contexts for the team. Programs
  should request the total number of simultaneous contexts to be created from
  the team during team creation. See Section~\ref{subsec:shmem_team_config_t}
  for more information on how to request contexts during team creation.

  A call to \FUNC{shmem\_team\_create\_ctx} on a team may fail, regardless
  of the configuration request for contexts, if the implementation is unable
  to create a context at the time when \FUNC{shmem\_team\_create\_ctx} is
  called.

  All explicitly created resources associated with a team must be destroyed
  before the \FUNC{shmem\_team\_destroy} routine is called. If a context
  returned from \FUNC{shmem\_team\_create\_ctx} is not explicitly
  destroyed before the team is destroyed, behavior is undefined.

  All \openshmem routines that operate on this context will do so with
  respect to the associated \ac{PE} team.
  That is, all point-to-point routines operating on this context will use
  team-relative \ac{PE} numbering.

  If \VAR{team} compares equal to \LibConstRef{SHMEM\_TEAM\_INVALID},
  then a nonzero value is returned and \VAR{ctx} is set to
  \LibConstRef{SHMEM\_CTX\_INVALID}.
  If \VAR{team} is otherwise invalid, the behavior is undefined.
}

\apireturnvalues{
  Zero on success and nonzero otherwise.
}

\begin{apiexamples}
    \apicexample
    {The following example demonstrates the use of contexts for multiple teams in a
    \CorCpp program. This example shows contexts being used to communicate within
    a team using team \ac{PE} numbers, and across teams using translated \ac{PE} numbers.}
    {./example_code/shmem_team_context.c}
    {}
\end{apiexamples}

\end{apidefinition}
