\apisummary{
    Concatenates blocks of data from multiple \acp{PE} to an array in every
    \ac{PE} participating in the collective routine.
}

\begin{apidefinition}

%% C11
\begin{C11synopsis}
int @\FuncDecl{shmem\_collect}@(shmem_team_t team, TYPE *dest, const TYPE *source, size_t nelems);
int @\FuncDecl{shmem\_fcollect}@(shmem_team_t team, TYPE *dest, const TYPE *source, size_t nelems);
\end{C11synopsis}
where \TYPE{} is one of the standard \ac{RMA} types specified by Table \ref{stdrmatypes}.

\begin{Csynopsis}
\end{Csynopsis}
\begin{CsynopsisCol}
int @\FuncDecl{shmem\_\FuncParam{TYPENAME}\_collect}@(shmem_team_t team, TYPE *dest, const TYPE *source, size_t nelems);
int @\FuncDecl{shmem\_\FuncParam{TYPENAME}\_fcollect}@(shmem_team_t team, TYPE *dest, const TYPE *source, size_t nelems);
\end{CsynopsisCol}
where \TYPE{} is one of the standard \ac{RMA} types and has a corresponding \TYPENAME{} specified by Table \ref{stdrmatypes}.

\begin{CsynopsisCol}
int @\FuncDecl{shmem\_collectmem}@(shmem_team_t team, void *dest, const void *source, size_t nelems);
int @\FuncDecl{shmem\_fcollectmem}@(shmem_team_t team, void *dest, const void *source, size_t nelems);
\end{CsynopsisCol}

\begin{DeprecateBlock}
\begin{CsynopsisCol}
void @\FuncDecl{shmem\_collect32}@(void *dest, const void *source, size_t nelems, int PE_start, int logPE_stride, int PE_size, long *pSync);
void @\FuncDecl{shmem\_collect64}@(void *dest, const void *source, size_t nelems, int PE_start, int logPE_stride, int PE_size, long *pSync);
void @\FuncDecl{shmem\_fcollect32}@(void *dest, const void *source, size_t nelems, int PE_start, int logPE_stride, int PE_size, long *pSync);
void @\FuncDecl{shmem\_fcollect64}@(void *dest, const void *source, size_t nelems, int PE_start, int logPE_stride, int PE_size, long *pSync);
\end{CsynopsisCol}
\end{DeprecateBlock}

\begin{apiarguments}

\apiargument{IN}{team}{A valid \openshmem team handle.}

\apiargument{OUT}{dest}{A symmetric array large enough
    to accept the concatenation of the \source{} arrays on all participating \acp{PE}.
    See table below in this description for allowable data types.}
\apiargument{IN}{source}{A symmetric data object that can be of any type permissible
    for the \dest{} argument.}
\apiargument{IN}{nelems}{The number of elements in the \source{} array. \VAR{nelems}
    must be of type \VAR{size\_t} for \Cstd.}

\begin{DeprecateBlock}
\apiargument{IN}{PE\_start}{The lowest \ac{PE} number of the active set of
    \acp{PE}.  \VAR{PE\_start} must be of type integer.}
\apiargument{IN}{logPE\_stride}{The log (base \CONST{2}) of the stride between
    consecutive \ac{PE} numbers in the active set. \VAR{logPE\_stride} must be of
    type integer.}
\apiargument{IN}{PE\_size}{The number of \acp{PE} in the active set. \VAR{PE\_size}
    must be of type integer.}
\apiargument{IN}{pSync}{
    A symmetric work array of size \CONST{SHMEM\_COLLECT\_SYNC\_SIZE}.
    In \CorCpp, \VAR{pSync} must be an array of elements of type \CTYPE{long}.
    Every element of this array must be initialized with the value
    \CONST{SHMEM\_SYNC\_VALUE} before any of the \acp{PE} in the active set
    enter \FUNC{shmem\_collect} or \FUNC{shmem\_fcollect}.}
\end{DeprecateBlock}

\end{apiarguments}

\apidescription{
    \openshmem \FUNC{collect} and \FUNC{fcollect} routines perform a collective
    operation to concatenate \VAR{nelems}
    data items from the \source{} array into the
    \dest{} array, over an \openshmem team or active set
    in processor number order. The resultant \dest{} array contains the contribution from
    \acp{PE} as follows:
    
    \begin{itemize}
    \item For an active set, the data from \ac{PE} \VAR{PE\_start} is first, then the
    contribution from \ac{PE} \VAR{PE\_start} + \VAR{PE\_stride} second, and so on.
    \item For a team, the data from \ac{PE} number \CONST{0} in the team is first, then the
    contribution from \ac{PE} \CONST{1} in the team, and so on.
    \end{itemize}
    
    The collected result is written to the \dest{} array for all \acp{PE}
    that participate in the operation. The same \dest{} and \source{}
    arrays must be passed by all \acp{PE} that participate in the operation.
    
    The \FUNC{fcollect} routines require that \VAR{nelems} be the same value in all
    participating \acp{PE}, while the \FUNC{collect} routines allow \VAR{nelems} to
    vary from \ac{PE} to \ac{PE}.

    Team-based collect routines operate over all \acp{PE} in the provided team argument. All
    \acp{PE} in the provided team must participate in the operation.

    Active-set-based collective routines operate over all \acp{PE} in the active set
    defined by the \VAR{PE\_start}, \VAR{logPE\_stride}, \VAR{PE\_size} triplet.
    As with all active-set-based collective routines,
    each of these routines assumes that
    only \acp{PE} in the active set call the routine. If a \ac{PE} not in the
    active set and calls this collective routine, the behavior is undefined.
    
    The values of arguments \VAR{PE\_start}, \VAR{logPE\_stride}, and \VAR{PE\_size}
    must be the same value on all \acp{PE} in the active set. The same
    \VAR{pSync} work array must be passed by all \acp{PE} in the active set.
    
    Upon return from a collective routine, the following are true for the local
    \ac{PE}:
    \begin{itemize}
    \item The \dest{} array is updated and the \source{} array may be safely reused. 
    \item For active-set-based collective routines, the values in the \VAR{pSync} array are
    restored to the original values.
    \end{itemize}
}


\apireturnvalues{
    Zero on successful local completion. Nonzero otherwise.
}

\apinotes{
    All \openshmem collective routines reset the values in \VAR{pSync} before they
    return, so a particular \VAR{pSync} buffer need only be initialized the first
    time it is used.

    The user must ensure that the \VAR{pSync} array is not being updated on any \ac{PE}
    in the active set while any of the \acp{PE} participate in processing of an
    \openshmem collective routine.  Be careful to avoid these situations: If the
    \VAR{pSync} array is initialized at run time, some type of synchronization is
    needed to ensure that all \acp{PE} in the working set have initialized
    \VAR{pSync} before any of them  enter an \openshmem routine called with the
    \VAR{pSync} synchronization array.  A \VAR{pSync} array can be reused on a
    subsequent \openshmem collective routine only if none of the \acp{PE} in the
    active set  are still processing a  prior \openshmem collective routine call
    that used the same \VAR{pSync} array.  In general, this may be ensured only by
    doing some type of synchronization.

    The collective routines operate on active \ac{PE} sets that have a
    non-power-of-two \VAR{PE\_size} with some performance degradation.  They operate
    with no performance degradation when \VAR{nelems} is a non-power-of-two value.
}

\begin{apiexamples}

\apicexample
    {The following \FUNC{shmem\_collect} example is for \CorCpp{} programs:}
    {./example_code/shmem_collect_example.c}
    {}

\end{apiexamples}

\end{apidefinition}
